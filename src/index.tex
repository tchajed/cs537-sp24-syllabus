\maketitle

\section*{Course info}

This course will introduce you to the broad field of operating systems.
Operating systems include a wide variety of functionality. This is an
introductory course and topics we will cover include basic operating system
structure, process and thread synchronization and concurrency, file systems and
storage servers, memory management techniques, process scheduling and resource
management, and virtualization.

\subsection*{Learning objectives}

At the end of this course, you will be able to:

\begin{itemize}[noitemsep]
  \item Explain the fundamental types of operating system abstractions including
        processes, synchronization, virtual memory and persistence.
  \item Design and implement system libraries and kernel calls, which are
        mechanisms provided to users to access and develop new operating-system
        functionality.
  \item Assess system performance and explain the impact of applying various
        algorithms and data structures to the complex operation of an operating system.
\end{itemize}

Course website: \url{https://pages.cs.wisc.edu/~oliphant/cs537-sp24} \\
Piazza: \url{https://piazza.com/class/lrl9gion4s33ro}

\subsection*{Instructors}
This class has two instructors: Louis Oliphant and Tej Chajed. We are teaching
the class jointly, even though you are technically assigned a section with one
of us as the instructor.

\subsection*{Lecture}
\begin{tabular}{llll}
  Section 2: & Science Hall 180 & Tuesday/Thursday & 9:30am--10:45am \\
  Section 1: & Social Sciences 5306 & Tuesday/Thursday & 1:00am--2:15pm
\end{tabular}

We suggest you attend the lecture you are assigned since the lecture rooms
cannot accomodate the students from both sections. Other than that, the sections
are the same (with shared content, assignments, and exams).

\subsection*{Instructor office hours}
\begin{itemize}
  \item Louis Oliphant: MWF 9--10am in CS 7358, or by appointment
  \item Tej Chajed: TBD in CS 7361, or by appointment
\end{itemize}

You can reach us by email at \texttt{ltoliphant@wisc.edu} and
\texttt{chajed@wisc.edu} but please use Piazza when possible.

\subsection*{TAs}
\begin{tabular}{ll}
Abigail Matthews & \texttt{amatthews5@wisc.edu} \\
John Shawger & \texttt{shawgerj@cs.wisc.edu} \\
Yurun Yuan & \texttt{yurun.yuan@wisc.edu} \\
Danial Saleem & \texttt{saleem5@wisc.edu} \\
Omid Rostamabadi & \texttt{omidrostamabadi@cs.wisc.edu} \\
Sunaina Krishnamoorthy & \texttt{skrishnamoo5@wisc.edu} \\
Fariha Islam & \texttt{fariha@cs.wisc.edu} \\
Aditya Das Sarma & \texttt{adassarma@cs.wisc.edu} \\
\end{tabular}

The TA and Peer Mentor office hours will be posted on Piazza and updated
regularly. TA office hours will typically be held in the CSL labs and basement
109.

\subsection*{Discussion sessions}
Discussion sections will be held on Wednesdays and will be led by the TAs. Please check your schedule for which
discussion section you should attend. The discussion section will cover topics that are related to the programming
projects and introduce other useful Unix/Linux tools.

\subsection*{Textbooks}

We will be using the \emph{free} OS textbook
\href{https://pages.cs.wisc.edu/~remzi/OSTEP}{\emph{Operating Systems: Three Easy
Pieces}}. You can also buy a printed copy if you like from the website.

For the programming projects, there are two textbooks that are recommended but not required
\begin{itemize}
\item \href{https://a.co/d/6cQipwF}{\emph{The C Programming Language} (2nd
ed.)}: A book written by the people who invented C
\item \href{https://a.co/d/9mCa0Gb}{\emph{Advanced Programming in the UNIX
Environment} (2nd ed.)}: This is a complete guide to programming in the Unix
environment and is useful if you want to become a Unix expert.
        \end{itemize}

\section*{Assignments \& Grading}

\subsection*{Programming assignments}

There will be around 7 programming assignments (projects) for this course whose
tentative schedule is given in the course website.

There are 4 slip days, 2 for the first 4 individual projects and 2 for the final
3 group projects. Students submit in programming projects on CSL machines. More
information about the same will be provided in the project specifications too.
Once you have used all your slip days, you will receive 100\% of points if turned
in on or before the deadline. If turned in late, 10 percentage points will be
deducted per day. You may only turn in assignments up to 3 days late.

\subsection*{Grading}
The course will have six main graded components: three exams, a set of
programming projects that will be spread out through the semester, lecture
quizzes, and a code review.

\begin{itemize}
  \item Programming projects: 50\%
  \item Midterm 1: 15\%
  \item Midterm 2: 10\%
  \item Midterm 3: 15\%
  \item Quizzes: 5\%
  \item Code review: 5\%
\end{itemize}

Assignments will be graded numerically, and those scores are combined at the end of the semester to form a letter
grade.
\begin{itemize}
  \item A: to receive an A, a student must consistently produce notable assignments in all aspects of the course.
  \item AB: to receive an AB, a student must consistently do acceptable work in all aspects of the course, and occasionally produce notable
        work.
  \item B: to receive a B, a student must consistently do acceptable work.
  \item BC: to receive a BC, a student must usually do acceptable work
  \item C: to receive a C, a student must usually make an effort at the assignment, and (at least) occasionally do acceptable work
\end{itemize}

Here is a numerical scale for grades. Based on difficulty of projects and exams, the actual cutoffs used may be lower
than these.

\begin{itemize}
  \item A: 93--100
  \item AB: 87--92.9
  \item B: 82--86.9
  \item BC: 78--81.9
  \item C: 70--77.9
  \item D: 60--69.9
  \item F: 0--59.9
\end{itemize}

\section*{Official university policies}

The university has standard, institution-wide syllabus text at
\url{https://guide.wisc.edu/courses/#syllabustext}.
